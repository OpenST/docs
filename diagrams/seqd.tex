\documentclass[landscape, 11pt, svgnames]{article}

\usepackage[showframe]{geometry}
\usepackage[T1]{fontenc}
\usepackage[utf8]{inputenc}
\usepackage[english]{babel}
\usepackage{listings}
\usepackage{tikz-uml}
\usepackage{amsfonts, amsmath, amsthm, amssymb}

\geometry{
  paperwidth=70cm,
  paperheight=100cm,
  margin=1cm
}

\date{\today}
\title{OpenST Protocol sequence diagrams beyond v0.9.2}
\author{Benjamin Bollen}

\lstdefinelanguage{tikzuml}{language=[LaTeX]TeX, classoffset=0, morekeywords={umlbasiccomponent, umlprovidedinterface, umlrequiredinterface, umldelegateconnector, umlassemblyconnector, umlVHVassemblyconnector, umlHVHassemblyconnector, umlnote, umlusecase, umlactor, umlinherit, umlassoc, umlVHextend, umlinclude, umlstateinitial, umlbasicstate, umltrans, umlstatefinal, umlVHtrans, umlHVtrans, umldatabase, umlmulti, umlobject, umlfpart, umlcreatecall, umlclass, umlvirt, umlunicompo, umlimport, umlaggreg}, keywordstyle=\color{DarkBlue}, classoffset=1, morekeywords={umlcomponent, umlsystem, umlstate, umlseqdiag, umlcall, umlcallself, umlfragment, umlpackage}, keywordstyle=\color{DarkRed}, classoffset=0,  sensitive=true, morecomment=[l]{\%}}

\begin{document}
\begin{minipage}[b]{0.55\linewidth}
\Huge \color{NavyBlue} \textbf{OpenST Protocol v0.9.3 } \color{Black}\\ % Title
\huge\textit{Proposed sequence diagrams for stake and mint - Benjamin Bollen, last edit \today}\\[1cm] % Subtitle
\end{minipage}

\begin{tikzpicture}
  \begin{umlseqdiag}
    \umlactor[class=Address]{Staker}
    \umlactor[class=Worker]{Staking Processor}
    \umlboundary[class=ERC20]{OST}
    \umlboundary[class=SK]{Branded Token Gate}
    \umlcontrol[class=SK]{OpenSTValue}
    \umlobject[class=SK]{SimpleStake}
    \umlobject[class=SK]{CoreUC}
    \umlcontrol[class=SK]{RegistrarVC}
    \umlboundary[class=Web3]{Value Chain}
    \umlactor[class=Worker, fill=purple!40]{Foundation}
    \umlboundary[class=Web3, fill=blue!40]{Utility Chain}
    \umlcontrol[class=SK, fill=blue!40]{RegistrarUC}
    \umlobject[class=SK, fill=blue!40]{CoreVC}
    \umlcontrol[class=SK, fill=blue!40]{OpenSTUtility}
    \umlboundary[class=ERC20, fill=blue!40]{Branded Token}
    \umlboundary[class=SK, fill=blue!40]{Token Holder}
  
    %%%
    %%%  Staker initiates stake
    %%% 
    
    % staker approves branded token gate contract on OST for amount
    \begin{umlcall}[op={: approve(gate, amount)}]{Staker}{OST}  
    \end{umlcall}
    % staker requests stake to gate
    \begin{umlcall}[dt=5, op={: stakeRequest(amount, tokenHolder)}]{Staker}{Branded Token Gate}
      % gate pulls amount from staker on OST
      \begin{umlcall}[fill=green!20, op={: transferFrom(staker, gate, amount)}, return=<<OST(amount)>>]{Branded Token Gate}{OST}
        % pull OST from Staking Processor
        \begin{umlcall}[dt=5, fill=green!20, type=return]{Staker}{OST}
        \end{umlcall}
      \end{umlcall}
      
      % emit StakeRequested event which Staking processor listens to
      \begin{umlcall}[dt=5, type=return, op={emit StakeRequested(staker, amount, tokenHolder)}]{Branded Token Gate}{Staking Processor}
      \end{umlcall}
    \end{umlcall}

    % Staking Processor evaluates request against policy (monetary and KYC/AML)
    \begin{umlfragment}[type=alt, label=accept, name=policy, inner xsep=2]
      % Staking Processor accepts request
      \begin{umlcall}[dt=6, op={: acceptRequest(staker, amount, hashLock)}]{Staking Processor}{Branded Token Gate}  
        % Gate calls on to OpenSTValue to stake (later abstract to library call)
        \begin{umlcall}[op={: stake(uuid, amount, hashLock, tokenHolder)}, return={nonce, unlockHeight}]{Branded Token Gate}{OpenSTValue}
          % OpenSTValue pulls amount plus bounty from Staking Processor to
          % its OST account balance
          \begin{umlcall}[dt=4, op={: transferFrom(processor, OpenSTValue, amount+bounty)}, fill=green!20, return=<<OST(amount+bounty)>>]{OpenSTValue}{OST}
            % pull OST from Staking Processor
            \begin{umlcall}[dt=10, fill=green!20, type=return]{Staking Processor}{OST}
            \end{umlcall}
          \end{umlcall}
          % store StakingIntentHash in contract storage
          \begin{umlcallself}[op={store StakingIntentHash},]{OpenSTValue}
          \end{umlcallself}
          % HTLC(processor, amount+bounty)
          \begin{umlcallself}[dt=0, op={HTLC(processor, amount+bounty)},]{OpenSTValue}
          \end{umlcallself}
          % emit StakingIntentDeclared
          \begin{umlcall}[type=return, op={emit StakingIntentDeclared(nonce, unlockHeight, StakingIntentHash)}]{OpenSTValue}{Staking Processor}
          \end{umlcall}
        \end{umlcall}
        % HTLC(staker, amount)
        \begin{umlcallself}[op={HTLC(staker, amount)}]{Branded Token Gate}
        \end{umlcallself}
      \end{umlcall}
      
      % Staking Processor rejects request
      \umlfpart[reject]
      \begin{umlcall}[dt=5, op={: rejectRequest(staker, amount)}]{Staking Processor}{Branded Token Gate}
        % transfer amount back to staker
        \begin{umlcall}[fill=green!20, op={: transfer(staker, amount)}]{Branded Token Gate}{OST}
          % return OST to staker
          \begin{umlcall}[type=return, fill=green!20, op=<<OST(amount)>>]{OST}{Staker}
          \end{umlcall}
        \end{umlcall}
      \end{umlcall}
    \end{umlfragment}
    \umlnote[x=2, y=-7]{policy}{evaluate request against policy (KYC/AML)}
    
    % optionally, staker can initiate revert after timeout and no action from Staking Processor
    \begin{umlfragment}[type=opt]
      % staker reverts Stake Request after timeout
      \begin{umlcall}[dt=12, op={: revertStakeRequest(staker, amount)}, fill=green!20]{Staker}{Branded Token Gate}
        % transfer amount back to staker
        \begin{umlcall}[fill=green!20, op={: transfer(staker, amount)}]{Branded Token Gate}{OST}
          % return OST to staker
          \begin{umlcall}[type=return, fill=green!20, op=<<OST(amount)>>]{OST}{Staker}
          \end{umlcall}
        \end{umlcall}
      \end{umlcall}
    \end{umlfragment}
    
    %%%
    %%%  OpenST Mosaic (Foundation reports respective state root of )
    %%%
    
    \begin{umlfragment}[type=loop, name=mosaic]
      
      \begin{umlcall}[dt=135, op={new block(blockHeight, stateRoot)}]{Utility Chain}{Foundation}
        % Foundation report state root of Utility chain on CoreUC on value chain
        \begin{umlcall}[op={: report(UC, blockHeight, stateRoot)}]{Foundation}{RegistrarVC}
          \begin{umlcall}[op={: commit(blockHeight, stateRoot)}]{RegistrarVC}{CoreUC}
            % state root of utility chain got reported on value chain
            \begin{umlcall}[dt=10, type=return, op={emit CommittedStateRoot(UC, blockHeight, stateRoot)}]{CoreUC}{Staking Processor}
            \end{umlcall}
          \end{umlcall}
        \end{umlcall}
      \end{umlcall}
          
      % Foundation report state root of Value chain on CoreVC on utility chain
      \begin{umlcall}[dt=165, op={new block(blockHeight, stateRoot)}]{Value Chain}{Foundation}
        \begin{umlcall}[op={: report(VC, blockHeight, stateRoot}]{Foundation}{RegistrarUC}
          % reporting through registrar is instant commit
          \begin{umlcall}[op={: commit(blockHeight, stateRoot)}]{RegistrarUC}{CoreVC}
            % state root of value chain got reported on utility chain
            \begin{umlcall}[dt=10, type=return, op={emit CommittedStateRoot(VC, blockHeight, stateRoot)}]{CoreVC}{Staking Processor}
            \end{umlcall}
          \end{umlcall}
        \end{umlcall}
      \end{umlcall}
      
    \end{umlfragment}
    \umlnote[x=28, y=-18]{mosaic}{Placeholder for OpenST Mosaic game}
    
    %%%
    %%% Staking Processor has observed a committed state root that includes the StakingIntentHash
    %%%
    
    % Staking Processor submits claim for StakingIntentHash by presenting Merkle proof
    \begin{umlcall}[dt=15, op={: claimStakingIntentHash(StakingIntentHash, merkleProof[], committedBlockHeight)}, return={emit ValidatedStakingIntentHash(StakingIntentHash)}]{Staking Processor}{OpenSTUtility}
      % OpenSTUtility checks StakingIntentHash against committed state root
      \begin{umlcall}[op={: getStateRoot(blockHeight)}, return={stateRoot @ blockHeight}]{OpenSTUtility}{CoreVC}
      \end{umlcall}
      % OpenSTUtility validate merkle proof
      \begin{umlcallself}[op={validate proof}]{OpenSTUtility}
      \end{umlcallself}
      % OpenSTUtility store valid StakingIntentHash
      \begin{umlcallself}[op={store valid StakingIntentHash}]{OpenSTUtility}
      \end{umlcallself}
    \end{umlcall}
    
    % Staking Processor submits pre-image data for StakingIntentHash
    \begin{umlcall}[dt=5, op={: ConfirmStakingIntent(StakingIntentHash, uuid, hashLock, amountST, amountUT, stakerProcessor, stakingProcessorNonce, unlockHeight)}, return={emit StakingIntentConfirmed(StakingIntentHash, nonce, unlockHeight, expirationHeight)}]{Staking Processor}{OpenSTUtility}
      % OpenSTUtility asserts valid pre-image data for StakingIntentHash
      \begin{umlcallself}[op={assert valid pre-image data}]{OpenSTUtility}
      \end{umlcallself}
      % OpenSTUtility checks StakingIntentHash against committed state root
      \begin{umlcall}[op={: getLatestHeight()}, return={latestHeight}]{OpenSTUtility}{CoreVC}
      \end{umlcall}
      % OpenSTUtility asserts grace period before unlockHeight
      \begin{umlcallself}[op={assert grace period before unlockHeight}]{OpenSTUtility}
      \end{umlcallself}
      % OpenSTUtility store mint object with StakingIntentHash and expiration Height
      \begin{umlcallself}[op={store mint with expirationHeight}]{OpenSTUtility}
      \end{umlcallself}
    \end{umlcall}
    
    %%%
    %%% Staking Processor has moved both value and utility chain to the first stage
    %%% and can now either proceed or revert by revealing the hash lock secret or await timeout
    %%%
    
    % 
    \begin{umlfragment}[type=alt, label=Proceed, name=phasetwo, inner xsep=6]
      \begin{umlcall}[dt=5]{Staking Processor}{OpenSTValue}
      \end{umlcall}
    \end{umlfragment}
    \umlnote[x=0, y=-45]{phasetwo}{With both value chain and utility chain configured correctly for StakingIntentHash, Staking Processor can proceed by revealing the unlock secret, or revert by awaiting the unlock height }
    
  \end{umlseqdiag}
\end{tikzpicture}


\end{document}

